\documentclass{article}
\usepackage{sivaSAFRANshort}
\chead{Assignment 1}
\begin{document}
	\begin{enumerate}
		\item
		Consider the following integral:
		$$I_n = \dint_0^1 x^{2n} \sin(\pi x) dx$$
		\begin{enumerate}
			\item
			Obtain a recurrence for $I_n$ in terms of $I_{n-1}$. (HINT: Integration by parts)
			\item
			Evaluate $I_0$ by hand
			\item
			Use the recurrence to obtain $I_n$ for $n \in \{1,2,\ldots,20\}$ in python.
			\item
			Obtain the integral using the builtin {\color{magenta}\textbf{scipy.integrate.quad}} command in python and compare the results to above.
			\item
			Explain your observation. We will see later in this course how to evaluate integrals to high accuracy.
		\end{enumerate}
		\item
		Recall the coffee-cooling problem discussed in class. In this assignment, let's assume that the cooling rate $r$ is not a constant but decays with time as $r(t) = \exp(-\alpha t^2)$. Take the initial temperature of the coffee-cup to be $85^{\circ}$C and the surrounding temperature to be $25^{\circ}$C. Explore the solution for values $\alpha \in \{0,1,2,3}$ per sq. minute. Note that $\alpha = 0$ should give us the solution for constant decay rate.
		\item
		Find the most accurate formula for the first derivative of the function $f$ at $x_i$ utilising known values of $f$ at $x_{i-1},x_i,x_{i+1}$ and $x_{i+2}$. The points are uniformly spaced. Give the leading error term and state the order of the method.
		\item
		A general Pad\'e type boundary scheme (at $i=0$) for the first derivative which doesn't alter the tridiagonal structure of the matrix can be written as
		$$f_0'+\alpha f_1' = \dfrac1h \bkt{af_0+bf_1+cf_2+df_3}$$
		\begin{enumerate}
			\item
			Obtain $a,b,c,d$ in terms of $\alpha$ so that the scheme is at least third-order accurate.
			\item
			Which $\alpha$ would you choose and why?
			\item
			Find all the coefficients so that the scheme would be fourth-order accurate.
		\end{enumerate}
		\item
		In numerical solution of boundary value problems in differential equations, we can sometimes use the physics of the problem not only to enforce the boundary conditions but also to maintain high-order accuracy near the boundary.
	\end{enumerate}
\end{document}