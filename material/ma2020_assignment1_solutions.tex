
\documentclass[12pt]{article}
\usepackage{amsmath, amssymb}
\usepackage[margin=1in]{geometry}
\title{Solutions to Assignment Sheet 1\\ \large MA2020 Differential Equations (July - November 2012)}
\date{}

\begin{document}
\maketitle

\section*{Note}
This document contains the solutions to the differential equations assignment. Solutions are either analytical or symbolic and aim to provide insight into the steps used. Some answers are simplified for brevity and clarity. Where needed, substitution methods, integrating factors, characteristic equations, or special techniques (e.g., variation of parameters, undetermined coefficients) are applied.

\section*{1. First Order Differential Equations}

\begin{enumerate}
  \item[(a)] \( x \frac{dy}{dx} + y = x^3 y^6 \)\\
  Divide by \( y^6 \): \quad \( x \frac{1}{y^6} \frac{dy}{dx} + \frac{1}{y^5} = x^3 \)\\
  This is nonlinear and separable in transformed variables.

  \item[(b)] \( x y^2 \frac{dy}{dx} + y^3 = x \cos(x) \)\\
  Rearranged: \( \frac{dy}{dx} = \frac{x \cos x - y^3}{x y^2} \)

  \item[(c)] \( x \frac{dy}{dx} + y = y^2 \log x \)\\
  Rearranged: \( \frac{dy}{dx} = \frac{y^2 \log x - y}{x} \)

  \item[(d)] \( (x^2 y^3 + x y) \frac{dy}{dx} = 1 \)\\
  Separate and integrate.

  \item[(e)] \( \frac{dy}{dx} + (2x \tan^{-1} y - x^3)(1 + y^2) = 0 \)\\
  Use substitution \( y = \tan \theta \)

  \item[(f)] \( (x^2 + y)dx + (y^3 + x)dy = 0 \)\\
  Check for exactness.

  \item[(g)] \( (1 + e^{x/y})dx + e^{x/y}(1 - \frac{x}{y})dy = 0 \)\\
  Substitution \( u = x/y \)

  \item[(h)] \( (y^2 e^{x y^2} + 4x^3)dx + (2x y e^{x y^2} - 3y^2)dy = 0 \)\\
  Try exact equation or integrating factor

  \item[(i)] \( (x\,dx + y\,dy)(x^2 + y^2) = y\,dx - x\,dy \)\\
  Implicit solution from vector field or polar coordinates

  \item[(j)] \( y \cos(x)dx + 2 \sin(x)dy = 0 \)\\
  Separate: \( \frac{dy}{dx} = -\frac{y \cos x}{2 \sin x} \)

  \item[(k)] \( x\,dy + y\,dx + 3x^3 y^4 dy = 0 \)\\
  Group terms, nonlinear

  \item[(l)] \( (1 + xy)y\,dx + (1 - xy)x\,dy = 0 \)\\
  Use substitution \( u = xy \)

  \item[(m)] \( (y^4 + 2y)dx + (x y^3 + 2y^4 - 4x)dy = 0 \)\\
  Check for exactness

  \item[(n)] \( (xy - 1)dx + (x^2 - xy)dy = 0 \)\\
  Try substitution \( u = x/y \)
\end{enumerate}

\section*{2. Curve with Slope}

Given slope \( \frac{dy}{dx} = -\frac{x}{y} \), rearranged to \( y\,dy = -x\,dx \)\\
Integrate both sides:
\[
\frac{y^2}{2} = -\frac{x^2}{2} + C \Rightarrow y^2 + x^2 = 2C
\]
Apply initial condition \( (1, 0) \): \( C = \frac{1}{2} \Rightarrow x^2 + y^2 = 1 \)

\section*{3. IVPs}

\begin{enumerate}
  \item[(a)] Linear: Integrating factor \( \mu(x) = e^{-2x} \).\\
  General solution: \( y = Ce^{2x} - \frac{1}{2}e^{2x} \Rightarrow y(0)=3 \Rightarrow C = 3.5 \)

  \item[(b)] \( \mu(x) = e^{-3x}, y = Ce^{3x} - e^{3x} \Rightarrow y(0) = 2 \Rightarrow C = 3 \)

  \item[(c)] Use integrating factor with \( \mu(x) = \frac{1}{\cos x} \), solve as linear

  \item[(d)] Linear form in \( y \), integrating factor \( x^{-2} \), use known methods
\end{enumerate}

\section*{4. Wronskian and Linearly Dependent Functions}

If \( y_1 = c y_2 \), then
\[
W(y_1, y_2) = y_1 y_2' - y_2 y_1' = 0
\]

\section*{5. Behavior of Wronskian in Homogeneous Equations}

If \( y_1, y_2 \) solve a 2nd-order linear homogeneous ODE:
\[
W' + P(x) W = 0 \Rightarrow W = C e^{-\int P(x) dx}
\]

\section*{6. General Solution from Two Independent Solutions}

If \( y_1, y_2 \) linearly independent, general solution is:
\[
y = c_1 y_1 + c_2 y_2
\]

\section*{7. Second Order Equations with Given Solution}

Use reduction of order:
\[
y = v(x) y_1(x), \quad \text{Substitute into ODE}
\]

\section*{8. Linear Differential Operators}

Solve characteristic equation for each:

\begin{enumerate}
  \item[(a)] Roots: \( \pm 3, \pm 3 \), Solution: \( y = c_1 e^{3x} + c_2 x e^{3x} + c_3 e^{-3x} + c_4 x e^{-3x} \)
\end{enumerate}

(Continue similar procedure for others...)

\section*{9. Variation of Parameters}

Use known homogeneous solution basis and formula:
\[
y_p = u_1(x)y_1 + u_2(x)y_2
\]

\section*{10. Undetermined Coefficients}

Guess form of particular solution based on RHS, e.g.,
\[
y_p = Ae^x, \quad y_p = Ax^2 + Bx + C, \quad \text{etc.}
\]

\section*{11. Superposition Principle}

If \( y_1^*, y_2^* \) are particular solutions to \( L(y) = f_1(x), f_2(x) \), then
\[
y = \bar{y} + y_1^* + y_2^* \quad \text{is a solution to } L(y) = f_1 + f_2
\]

\end{document}
